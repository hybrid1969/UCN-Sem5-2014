\label{ProblemAnalysis}
\section{Problem Analysis}

In most games there is often a goal within the game that leads to completion. In most cases these goals are to get from one end of a level to the other, or explore the world to find certain points of interest. However level and/or world design can often be a long, tiresome and expensive process to go through when creating games, and can take weeks, month or even years to create, such that it matches the game's theme.

With procedural content generation we can accomplish dynamic generation of game worlds in a few minutes, but also a large variety of worlds, so the game will always look new and different on a new game start. However it is all dependent on the algorithm(s) generating the world, and an algorithm can therefore require a lot of work and fine tuning to produce a result which is acceptable.

Even though an algorithm can create unique and acceptable game world, a lot of game worlds are still manually created. One of the reasons is so the game developers are in complete control over the game. It seems that manual created levels are preferred when there is speaking of a game where the world should be created specific around the game, where a game with procedural generated content often seems to have been created around the procedural content generation.

The reason we have decided to look into procedural content generation is due to it's huge potentials of creating almost endless and randomized content to the users, but also due to it may save a game development team lots of money on level designer, either by creating everything procedural, or parts of it and only have a designer optimize the generated result.

%An other option could be to combine the two ways of creating levels, so that a world can be generated, and then worked on by the designers to look exactly as they want it to, but it does look a lot like double up on work.
% as such game can not have real objective or a deep storyline, where some triple A titles such as the Walking Dead or the Last of us where everything is manually created tells a deep story that you play through.
%TODO: discuss sources
\section{Manually versus Procedural Created Content}

With that amount of work one would think that most game universes are big, but truth is they are not. A lot of game universes are very small. For instance, Azeroth which is the world in World of Warcraft, is suggested to only be 80 Miles$^2$ or 207,20 km$^2$ \cite{GameWorldSize} which is roughly 208 times smaller than Denmark. I do not here by say World of Warcraft would be a better game if the world was larger but simply using it for comparison.

An other thing to note about manually created content is that it is often the same experience each time you play the game. This can make the replay-ability worse, especially if you have already seen everything and know the world well. However some games can have a dynamic story line that changes depending of your choices or/and have character customization and that way improve the replay-ability of the game.

We have now pointed out some key flows about static created content, however procedural generated content does also have its flaws. There is a lot of places where advantages and disadvantages are discussed. Some of the disadvantages we see with procedural content generation includes that the world can be unplayable, here by meaning that a player could be stuck with no way of proceeding in the game, which can lead to huge annoyances. An other huge problem can be the variety in the content where some levels can be generated such that a player can be extremely powerful after playing for few minutes, and other levels can end in the player barely finding any items and there by being very weak.

Opposite to manually created level procedural content generation can offer huge if not infinite levels and worlds but the world will also be different almost every time a new game is started and that can improve the replay-ability a lot of the game. With so huge worlds variety could be an important factor as well. If a game world is non stop desert, non stop mountains, or generally the world look the same, the game can quickly feel monotone and dull for the players.

Other reasons for using procedural content generation could be to save memory, as levels often can use up a lot of space. But for procedural content generation the only memory needed is for the code, as the code behind will generate the world, but not necessarily save it to the memory. A more philosophical way of looking at a reason could be that a computer algorithm could produce results that level designers can never have thought of, or imagined could be made by hand, and in that way create something truly unique.